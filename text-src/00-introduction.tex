\chapter*{Introduction}

Since the beginning of computers people imagined how artificial intelligence (AI) will either help us with everyday struggles, or take over the world one day and destroy us. Today, it is becoming a part of our everyday life, without us even knowing sometimes. We have multiple fields of AI science like expert systems, simple machine learning or advanced neural networks (NN).

A neural networks experience nowadays great development for another time in history. Thanks to stronger hardware and bigger computational capabilities, deep learning became feasible. With this achievement, we are finally able to use NN for purposes in real world. NNs help with predicting diseases, driving cars and they can become fast learners in playing video games.

First part of the thesis researches and examines field of Artificial Neural Networks (ANN) and Spiking Neural Networks (SNN). Current state of neural components like neurons and synapses, algorithms for learning and toolkits will be described. Some advantages and disadvantages will be compared together with differences between ANNs and SNNs.

Goal of this work is to design both ANN and SNN that will classify electroencephalography (EEG) data. Both approaches will be tested and compared according to their accuracy. Data were acquired from people, who were told to think about a number and then their brainwaves were recorded. This work will find out results of different classification approaches with both ANNs and SNNs.

In chapter 1 there is thorough description of both network models with its components. Chapter 2 focuses on training of each network with both supervised and unsupervised learning. Chapter 3 goes over simulation tools for designing both types of networks, including those used for experiment. Chapter 4 introduces neuromorphic hardware including a new component in electronics called memristor. Chapter 5 and following are dedicated to the experiment and results, which are final goal of this work.

